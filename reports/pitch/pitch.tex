\documentclass[a4paper,10pt]{article}
\usepackage[round]{natbib}
\usepackage{hyperref}

\title{Pitch}
\author{Anton Mulder \and Xeryus Stokkel \and Angelo Karountzos \and Ren\'e
    Mellema (s2348802)}
\date{}

\begin{document}

\maketitle

\subsection*{Problem}
In both the digital and the real world there is a lot of information.
Ideally, we would want to use both these sources of information at the same
time. At the moment there is no real good way to do that, but we believe
that the framework of multi-agent systems can be used as a perfect bridge
between these two worlds. 

\subsection*{State of the Art}
This was done in \citet{intframe}. Here a robot was directed towards
a target location using an external path planning service which was
supposed to be accessed over the internet. 

Since we want to imporve on this in the context of Swarm robotics, we also
found \citet{selforg}, were two kinds of robots were used to plan a path
through an obstructed environment. 

\subsection*{New idea}
Our new idea is to incorporate these two ideas to build a simulation that uses
smaller search robots to collect information about a maze. This information will
then be handed over to the path planning agents to build a path for larger
rescue robots to go into the maze and collect ``survivors''. The whole
process will be co-ordinated by a supervisor. 

This expands on the ideas of
\citet{intframe} by more closely incorporating real world and digital
information and services. It will expand on \citet{selforg} because it is a
different way to make a heterogeneous group of robots.

\subsection*{Results}
The result will be a collaborating team of heterogeneous robots that
combine real world information and digital information. 

\subsection*{Relevance}
This is relevant because better search and rescue techniques will be able
to save more people faster. It should also tell us more about incorperating
information in the real and digital worlds. 

\bibliographystyle{plainnat}
\bibliography{references}

\end{document}
