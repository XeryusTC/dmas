\documentclass[a4paper,10pt]{article}
\usepackage[round]{natbib}
\usepackage{hyperref}

\title{Pitch}
\author{Anton Mulder \and Xeryus Stokkel \and Angelo Karountzos \and Ren\'e
    Mellema (s2348802)}
\date{}

\begin{document}

\maketitle

\begin{itemize}
    \item[Problem] Incorporating real world information and internet
        services.
    \item[State of the Art] \citet{intframe} and \citet{selforg}.
    \item[New Idea] Collection of information by a swarm of robots to guide
        a larger robot through a maze
    \item[Results] A collaborating team of robots
    \item[Relevance] Using a swarm for search and rescue
\end{itemize}

\section*{Problem}
In both the digital and the real world there is a lot of information.
Ideally, we would want to use both these sources of information at the same
time. At the moment there is no real good way to do that, but we believe
that the framework of multi-agent systems can be used as a perfect bridge
between these two worlds. 

\section*{State of the Art}
This was done in \citet{intframe}. Here a robot was directed towards
a target location using an external path planning service. We want to
expand upon this idea by using a swarm to collect information that the path
planning algorithm can use.

This is similar to the research done in \citet{selforg}, were two kinds of
robots were used to plan a path through an obstructed environment. The
difference between this research and our research is that in
\citet{selforg} there was a large focus on self organisation, while we will
build a more organized system to incorporate elements from the digital
world. 

\section*{New idea}
Our new idea is to incorporate these two ideas to build a simulation that uses
smaller search robots to collect information about a maze. This information will
then be handed over to the path planning agents to build a path for larger
rescue robots to go into the maze and collect ``survivors''. The whole
process will be co-ordinated by a supervisor. 

This expands on the ideas of
\citet{intframe} by more closely incorporating real world and digital
information and services. It will expand on \citet{selforg} because it is a
different way to make a heterogeneous group of robots.

\section*{Results}
The result will be a collaborating team of heterogeneous robots that
combine real world information and digital information. 

\section*{Relevance}
This is relevant because better search and rescue techniques will be able
to save more people faster. It should also tell us more about incorperating
information in the real and digital worlds. 

\bibliographystyle{plainnat}
\bibliography{references}

\end{document}
