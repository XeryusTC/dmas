\documentclass{article}
\usepackage{bnaic}
\usepackage{hyperref}

\title{\textbf{\huge Project Report}}
\author{Anton Mulder \and Xeryus Stokkel \and Angelo Karountzos \and Ren\'e
    Mellema}
\date{University of Groningen}

\pagestyle{empty}

\begin{document}

\maketitle

\begin{abstract}
    \noindent
\end{abstract}

\section{Introduction}
In modern times, a lot of information is spread out among many sources.
One of these sources is the internet, where a large collection of 
\subsection{Problem}
\subsection{State of the Art}
\cite{intframe}.
\subsection{New Idea}

\section{Method}
\subsection{Simulation Model}
Our simulation consists of two web services, two soft bots and a supervisor.
The web services are a data store and a path planner. The data store is used
to store the map, while the path planner can be given a map, a starting
point and a target location. The soft bots are the searchers and the
rescuers. The searchers explore the maze with help from the supervisor,
while the rescuers are sent into the maze to collect the targets. The
supervisor is the link between all these components. It gets the data about
the map from the searchers and forwards it to the data store. 

When a searcher gets stuck, the supervisor requests the current map. It
then gives the map, the location of the stuck searcher and a position that
has not been visited yet to the path planner. The planner will return a
path if it is possible. The supervisor then sends this path to the searcher
so it can resume its quest.
\subsection{Experimental design}

\section{Results}
\subsection{Experimental Findings}
\subsection{Interpretation}

\section{Conclusion}
\subsection{Discussion}
\subsection{Relevance}

\bibliographystyle{plain}
\bibliography{references}
\end{document}
