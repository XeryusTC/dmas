In our experiment we did not find the difference we had expected. This is most
likely because we did not stress the communications enough. When more agents
are added there might be a difference between the different supervisors, but
our current test machine would then become the bottleneck. Another method of
trying to stress communication more is to use a larger maze, but running the
simulation would then take a prohibitively long time. Another way to try
to find this result would be to build a simulation where communication is of
more importance than in our current simulation. In such a simulation, the
communication bottleneck would show itself earlier. 

\subsection{Discussion}

One important difference between the monolithic and the distributed approach is
the ease of implementation. The monolithic mothership is complicated but all
other agents are relatively simple. This makes it hard to implement the
mothership itself but the agents for the robots and services are trivial to
implement after their core functionality is implemented. The mothership itself
becomes quite complicated as there is quite a lot of bookkeeping involved to
keep track of which responses need to be send to which agents. For example,
when an agent needs to be send somewhere then a path to that location should
be planned which is done by a separate agent. Once this path finding agent
responds with a path then the mothership needs to forward it on to the right
agent. Because multiple paths can be planned at the same time it needs to keep
track of which path needs to be forwarded to the which agent. This kind of
bookkeeping is what makes the monolithic supervisor complicated.

The distributed supervisor method is harder to implement because more
communication is needed between the different agents. This means that there are
more moving parts, but they are also smaller than with the monolithic
mothership. More care needs to be taken to implement this and make all the
agents communicate with each other correctly. There is also the complication
that agents need to send confirmation to the supervisor whenever they receive
an order, it is sometimes possible that they are not able to execute this order
and thus they need to notify the supervisor to issue it again later.

\subsection{Relevance}

With our system we got a better insight on how to incorporate information from
the digital and real world and how to combine both information sources in a
multi-agent framework. These insights could help on improving search and rescue
techniques by utilising information that is also available on the web or other
resources like digital maps or satellite images for larger areas. Incorperating
these kind of maps could lead to more effective search in rescue in large areas
like after the earthquakes in Ha\"iti or Tibet. This method can go both ways:
the external resources can be used to plan where search and rescue robots need
to be deployed like in our simulation. The system can also annotate external
data so that other teams know where to concentrate their efforts, like
indicating points of interests on shared maps.

The system can also be applied to other situations, like traffic flow
coordination or in distribution centres where multiple robots drive around and
gather different objects to be send out to different facilities or costumers.
The applications for this technology are endless, especially in domains where
cooperation between different teams is necessary. Sharing data and computation
over the internet means that vital resources, including development time can be
saved. Sharing information, strategic plans and methods between different teams
helps to free up resources that would otherwise be needed multiple times
because each team has their own (incompatible) version.
