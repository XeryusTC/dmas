In modern times, a lot of information is spread out among many sources. One
of these sources is the internet, where a large collection of different
entities contain both know-what and know-how.  A different source is the
real world, where robots can collect information by sensing their
environment. We would want to combine these sources to get the most up to
date and complete information about the world. This could for example be
relevant in a situation where we want to collect information about traffic
flow so navigation systems can try to control the flow.

\subsection{Problem description} 

A global system where these separate systems can communicate and share information
is not implemented yet. A multi-agent system would be a good platform to
build such a system on, since it allows services to negotiate terms for
co-operation and advertise their services. 

%In this particular paper we are describing
%a situation where various robots work together under a common coordinator.
%In our case that goal is to retrieve people who have been trapped in a
%maze. To this end, many different situations have to be taken into account
%for the smooth extraction of the victim, such as robot collision avoidance,
%robot destruction or prioritization between targets.  Another major pitfall
%is the computation of all the data available. For a small area, a regular
%desktop power machine would be sufficient, but for larger real world
%applications it would not scale up properly because there are many machines
%involved, including some physical robots. In that case the agents should be
%able to outsource the computation to other agents which run on hardware
%that is dedicated for computing the task.

\subsection{State of the Art} 

Such a system has been implemented before in \cite{intframe}. This system
consisted of multiple physical agents, one web service that did path
planning and a supervisor that coordinated the interaction between the two.
Since there was only one web service to chose from, the system did not do
any form of negotiation or service discovery yet. They also used one
ontology for all communication and information.

\subsection{New Idea} 

In \cite{intframe} the supervisor agent was a monolithic agent, while its
tasks could be separated into multiple agents. This would have as downside
that this becomes a single bottleneck for communication, since it is the
component that is guiding all communication. The upside of this system is
that one agent has all the state needed to coordinate the rest. We do
suspect that this upside does not outweigh the downsides and want to test
this.

In order to test this, we will create both a monolithic supervisor, the
mothership; and a distributed agent, where all the responsibilities are
divided among more components. This will allow us to test these two
different architectures. In the case of the distributed agent most
communication will go from the robots to special forwarders that
communicate with the web services. This means that agents will only communicate
with the supervisor when they need services it is responsible for, relieving
the stress from this part of the system.

Another change is that the services discoverable instead of hard coding them
into the agent, although we will still only have one instance of each web
service so this wouldn't strictly be necessary. A third change that we want to
make is in the environment. In \cite{intframe} the environment was simplistic,
whereas we want to use a more complicated maze. This will also increase the
stress on the system, making it easier to test the different architectures
for the supervisor. 

% The main procedure is as follows; first search robots
% are sent into the uncharted maze. As they move around the immediate
% environment is sensed and this information gets stored in a central
% database. The management of this database is the task of a different agent.
% This means that creating a global map is at least one agent-to-agent
% interaction. There is also a general supervising agent that can tell the
% search robots where to go in the maze so that exploration is optimized and
% robots don't explore the corridors that have already been explored by other
% agents. This supervising agent can be just focus on this task, but it can
% also be a stricter manager which handles most of the communication between
% different agents, especially between the robot agents and any other agents.
% This means that this managing agent ties all the different services
% together in a single monolithic agent. This is the system that was used in
% \cite{intframe}.
% 
% We plan to compare the usage of a monolithic agent versus the distributed
% system. In the distributed system the robot agents can communicate directly
% with the agents that communicate with web services. In this case the
% supervisor will do the minimum amount necessary to create a global plan for
% the robotic agents but these will have to figure out the details and
% communicate with services themselves. We hope that this will remove the
% single point of failure that a monolithic agent has.
